\documentclass[../main.tex]{subfiles}



\begin{document}



\chapter*{Introducció}

Al present document trobareu uns apunts de l'assignatura curricular obligatòria ``Topologia i Geometria Global de Superfícies'', cursada per mi al semestre de tardor de 2020, a la Universitat de Barcelona, Facultat de Matemàtiques i Informàtica. Aquesta assignatura ha estat impartida pel Dr. Vicente Navarro, que va començar de manera semipresencial (1h setmanal presencial de resoldre problemes i 3h no presencials), i va acabar de manera completament online i, per tant, autodidacta perquè el professor no m'agradava gens.

La meva bibliografia principal seran els documents que ha anat penjat el professor a mesura que avançava el curs al campus virtual. Aquests documents són els substituts, per dir-ho d'alguna manera, de les classes presencials de teoria en les quals el professor apunta a la pissarra i jo prenc apunts. Així doncs, la majoria de casos, copiaré les definicions, proposicions, etc. dels mateixos apunts. 

Una altra font principal serà la pàgina web de  \texttt{mathonline.wikidot.org} \cite{mathonline} que ja vaig utilitzar per l'assignatura de topologia. Aquesta pàgina, però, conté detallada informació i moltíssims exemples de la part de topologia únicament. És una pàgina que em va servir molt en el seu moment per veure il·lustracions del que s'està fent, així com nombrosos exemples.

Aquesta és una assignatura que és recomanable cursar després d'haver cursat Topologia. La primera part, com ja he esmentat, és una continuació d'un curs en Topologia i per tant partiré dels meus apunts de l'assignatura de Topologia.



\subfile{chapters/homotopia.tex}
\subfile{chapters/grup_fonamental.tex}
\subfile{chapters/superficies_topologiques.tex}













\end{document}