\documentclass[../main.tex]{subfiles}






\begin{document}



\section{Actions and Sylow theorems}

\begin{defi}
[Action of a group on a set]\index{Action of a group on a set} Let $X$ be a set and $G$ a group. A map
\begin{equation}
    \notag
    \begin{array}{rl}
        \rho:G\times X & \longrightarrow X \\
        (g,x) & \longmapsto \rho(g,x):=gx
    \end{array}
\end{equation}
is called an \textit{action from $G$ on $X$}, or alternatively it is said that \textit{$G$ acts on $X$} if it satisfies the following conditions:
\begin{enumerate}[(i)]
    \item $ex = x$, for all $x\in X$;
    \item $g(g'x) = (gg')x$, for all pair $g,g'\in G$ and all element $x\in X$.
\end{enumerate}
\end{defi}

If $G$ acts on a set $X$, we also say that $X$ is a $G$-set.

If a group $G$ acts on a set $X$, then fixing the first variable, say $g$, gives a function $\alpha_g:X\rightarrow X$, namely $\alpha_g:x\mapsto gx$. This function is a permutation of $X$, for its inverse is $\alpha_{g^{-1}}$:
\begin{equation}
    \notag
    \alpha_g\alpha_{g^{-1}} = \alpha_1 = 1_X = \alpha_{g^{-1}}\alpha_g
\end{equation}

\begin{defi}
[Orbit, stabilizer, orbit space]\index{Orbit}\index{Stabilizer}\index{Orbit space} If $G$ acts on $X$ and $x\in X$, then the \textit{orbit} of $x$, denoted by $\mathcal{O}(x)$, is the subset
\begin{equation}
    \notag
    \mathcal{O}(x) := \{gx\;:\;g\in G\}\subseteq X;
\end{equation}
the \textit{stabilizer} of $x$, denoted by $G_x$ or also $E(x)$, is the subgroup
\begin{equation}
    \notag
    G_x:=\{g\in G\;:\;gx = x\}\subseteq G.
\end{equation}
The \textit{orbit space}, denoted by $X/G$, is the set of all the orbits.
\end{defi}

If $G$ acts on a set $X$, define a relation on $X$ by $x\sim y$ if, an only if there exists $g\in G$ with $y = gx$. It is easy to see that this is an equivalence relation whose equivalence classes are the orbits. The orbit space is the family of equivalent classes.

Before getting into the heavy examples, let's see a few properties that will be useful in the future.

\begin{prop}
Let $\rho$ be an action of $G$ over $X$. Then, 
\begin{enumerate}[(i)]
    \item if $x\in X$ is an arbitrary element, the stabilizer $G_x$ is, indeed, a subgroup of $G$;
    \item it is satisfied $gG_xg^{-1} = G_{gx}$, for all $x\in X$ and $g\in G$.
    \item if $G$ is a finite group and $x\in X$, then $\#Gx$ is finite and, moreover,
    \begin{equation}
        \notag
        |Gx| = [G:G_x] = \frac{|G|}{|G_x|}.
    \end{equation}
\end{enumerate}
\end{prop}
\begin{proof}
\noproof
\end{proof}\index{$G_x$}



\begin{ej}[Heavy examples]\index{Heavy examples of actions}\label{ej:heavyexamplesactions}
Let's begin with the heavy examples. I call them ``heavy examples'' because they are a little bit more than examples, as they work for the theory as well. And they are heavy, because they are a lot to take.
\begin{enumerate}[(1)]
    \item Given a group $G$, let's consider the action
    \begin{equation}
        \notag
        \begin{array}{rl}
            \rho:G\times G & \longrightarrow G \\
            (g,h) & \longmapsto ghg^{-1}
        \end{array}
    \end{equation}
    this action is called \textit{action by conjugation on a group by himself}\index{Action by conjugation on a group by himself}. We define the \textit{centre of $G$}\index{Centre of $G$} as the kernel of $\rho$, i.e.
    \begin{equation}
        \notag
        Z(G) = \{g\in G\;:\;ghg^{-1} = h,\;\forall h\in G\}
    \end{equation}
    Notice that it is the same definition we got at the beginning. We also say that the orbit of an element $h\in G$ by this action is the \textit{conjugation class of $h$}\index{Conjugation class}. Finally, we call \textit{centralizer of $h$ in $G$}\index{Centralizer} to the stabilizer of an element by this particular action. It is written
    \begin{equation}
        \notag
        Z_G(h):=G_x = \{g\in G\;:\;ghg^{-1} = h\}
    \end{equation}\index{$Z_G$}
    
    \item If $G$ is a group, $H$ a subgroup of $G$ and $g\in G$, we call \textit{conjugated of $H$ by $g$}\index{Subgroup conjugated by an element} to the set $gHg^{-1} = \{ghg^{-1}\;:\;h\in H\}$. It can be easily proved that it is indeed a subgroup of $G$. Take now $\mathcal{H}$ the set of all subgroups of $G$, i.e.\index{$\mathcal{H}$}
    \begin{equation}
        \notag
        \mathcal{H}:= \{H\subseteq G\;:\;\text{$H$ subgroup of $G$}\}
    \end{equation}
    Then we consider the following map
    \begin{equation}
        \notag
        \begin{array}{rl}
            \rho:G\times\mathcal{H} & \longrightarrow \mathcal{H} \\
            (g,H) & \longmapsto gH_0g^{-1}
        \end{array}
    \end{equation}
    We can observe that $\rho$ is an action of $G$ on $\mathcal{H}$, the set of all subgroups of $G$, and we can call it \textit{action by conjugation of a group on its subgroups' set}\index{Action by conjugation of a group on its subgroups' set} The orbit of a picked subgroup $H_0$ by this action is the set of all its conjugated, i.e.
    \begin{equation}
        \notag
        \mathcal{O}(H_0) = \{gH_0g^{-1}\;:\;g\in G\}
    \end{equation}
    The stabilizer of $H_0\in\mathcal{H}$ by this action is called \textit{normalizer of $H_0$ in $G$}\index{Normalizer}\index{$N_G(H_0)$} and it is written $N_G(H_0)$. It has been proved in ``Estructures Algebraiques'' that $N_G(H_0)$ is the biggest subgroup of $G$ which contains $H_0$ as a normal subgroup.
    
    \item Let $H$ be a subgroup of a group $G$, then we consider the action 
    \begin{equation}
        \notag
        \begin{array}{rl}
            \rho:H\times G & \longrightarrow G \\
            (h,g) & \longmapsto hg
        \end{array}
    \end{equation}
    which is called the \textit{right translation}. We can analogously consider the \textit{left translation}. This allows us to define the \textit{centralizer} of a subgroup $H$ of $G$ as
    \begin{equation}
        \notag
        C_G(H) = \{g\in G\;:\;gh = hg,\;\forall h\in H\}
    \end{equation}
    which is easy to see that it is a subgroup of $G$ and also that $C_G(G) = Z(G)$. Finally, note that $C_G(H)\subseteq N_G(H)$
\end{enumerate}
\end{ej}



\begin{prop}
[The $N/C$ lemma]\index{$N/C$ lemma} 
\begin{enumerate}[(i)]
    \item If $H\subseteq G$, then $C_G(H)\vartriangleleft N_G(H)$ and there is an embedding
    \begin{equation}
        \notag
        N_G(H)/C_G(H)\hookrightarrow \mathrm{Aut}(G).   
    \end{equation}
    \item $G/Z(G)\cong \mathrm{Inn}(G)$, where $\mathrm{Inn}(G)$\index{$\mathrm{Inn}(G)$} is the subgroup of $\mathrm{Aut}(G)$ consisting of all the inner automorphisms\footnote{An \textit{inner automorphism} is an automorphism of a group $G$, given by the conjugated action. That is, all homomorphisms $f:G\rightarrow G$ such that, for some fixed $a\in G$ we have for all $x\in G$, $f(x) = axa^{-1}$.}.
\end{enumerate}
\end{prop}
\begin{proof}
\noproof
\end{proof}

\begin{nota}
\index{Outer automorphisms} We claim that $\mathrm{Inn}(G)\vartriangleleft\mathrm{Aut}(G)$. Indeed, if $\varphi\in\mathrm{Aut}(G)$ and $g\in G$, then 
\begin{equation}
    \notag
    \varphi\gamma_a\varphi^{-1}:g\mapsto\varphi^{-1}g\mapsto a\varphi^{-1}ga^{-1}\mapsto\varphi(a)g\varphi(a^{-1})
\end{equation}
thus $\varphi\gamma_a\varphi^{-1} = \gamma_{\varphi(a)}\in\mathrm{Inn}(G)$. Recall that an automorphism is called \textit{outer} if it is not inner. The \textit{outer automorphism group} is defined by $\mathrm{Out}(G) = \mathrm{Aut}(G)/\mathrm{Inn}(G)$.
\end{nota}

We may now define, given a group $G$ and an action $\rho$ of $G$ over a set $X$, the relation
\begin{equation}
    \notag
    x\sim y\Longleftrightarrow y = gx,
\end{equation}
for some $g\in G$, where $x,y\in X$. This is an equivalence relation which is very easy to prove. We may observe that the orbits of all $x\in X$ form a partition of $X$, i.e. $\bigcup_{x\in X}\mathcal{O}(x) = X$. We can write $X/G$ to express $G/\sim$, i.e. the set of all the orbits. 

\begin{prop}
It is satisfied
\begin{equation}
    \notag
    |X| = \sum_{\mathcal{O}\in X/G}|\mathcal{O}| = \sum_{x\in X}|\mathcal{O}(x)|
\end{equation}
and if $X$ is finite, say $X = \{x_1,\ldots,x_\ell\}$, then
\begin{equation}
    \notag
    |X| = \sum_{i=1}^\ell |\mathcal{O}(x_i)|
\end{equation}
\end{prop}
\begin{proof}
\noproof
\end{proof}

\begin{ter}
If $G$ acts on a set $X$ and $x\in X$, then
\begin{equation}
    \notag
    |\mathcal{O}(x)| = [G:G_x],
\end{equation}
the index of the stabilizer $G_x$ in $G$.
\end{ter}
\begin{proof}
\noproof
\end{proof}

With this result, we can rewrite the last proposition we saw by
\begin{equation}
    \notag
    |X| = \sum_{x\in X}[G:G_x]
\end{equation}
All these equations are called \textit{orbit equations}\index{Orbit equations}. We can adapt the orbit equations to the heavy example's actions. For example, if $x$ lies in a finite group $G$, then the number of conjugates of $x$ is the index of its centralizer; or if $H$ is a subgroup of a finite group $G$, then the number of conjugates of $H$ in $G$ is $[G:N_G(H)]$.

We will now introduce the $p$-groups, after saying some more results of orbits and actions, and finally we will state Cauchy's Theorem which characterizes the subgroups of a finite group regarding the divisors of its order.

\begin{defi}
[$p$-group]\index{$p$-group} Let $p$ be an odd prime number and $G$ a finite group. Then $G$ is called \textit{$p$-group} if there exists $r\in\mathbb{Z}_{\geq 0}$ such that $|G| = p^r$. Analogously we will say that $H$ is a \textit{$p$-subgroup}\index{$p$-subgroup}.
\end{defi}


\begin{prop}
\begin{enumerate}[(1)]
    \item If $G$ is a $p$-group that acts on a finite set $X$, then
    \begin{equation}
        \notag
        |X|\cong |X_0|\quad \text{(mod $p$)}
    \end{equation}
    where $X_0$ is the subset of $X$ of fixed points.
    \item If $G$ is a $p$-group, its center $Z(G)$ is not trivial.
    \item If $H$ is a $p$-subgroup of a finite group $G$, then
    \begin{equation}
        \notag
        [N_G(H):H]\equiv[G:H]\quad\text{(mod $p$)}
    \end{equation}
\end{enumerate}
\end{prop}
\begin{proof}
\noproof
\end{proof}

We can finally state the Cauchy's theorem, although it will not be proved in this notes, because again, it is not very relevant for the purpose of this text.

\begin{ter}
[Cauchy] Let $G$ be a finite group of order $m$ and $p\in\mathbb{Z}$ prime which divides $m$. Then $G$ has an element (and hence a subgroup) of order $p$.
\end{ter}

I will now state some results regarding simple groups, due to the fact that I will use them in the future and I have not yet even defined them.

\begin{defi}
[Simple group]\index{Simple group} A group $G$ is called \textit{simple} if $G\not=\{1\}$ and $G$ has no normal subgroups other than $\{1\}$ and $G$ itself.
\end{defi}

\begin{prop}
An abelian group $G$ is simple if and only if it is finite of prime order.
\end{prop}
\begin{proof}
\noproof
\end{proof}

\begin{coro}
A finite $p$-groupe $G$ is simple if and only if $|G| = p$.
\end{coro}

From this point I will start with Sylow Theorems. I will not only state these theorems, but also I will give relevant results and definitions around them. We begin by defining a new type of group.

\begin{defi}
[Sylow $p$-subgroup]\index{Sylow $p$-subgroup} Let $G$ be a finite group and $P$ a subgroup. We call $P$ a \textit{Sylow $p$-subgroup} of $G$ if $P$ is a $p$-subgroup and also it is maximal between all possible $p$-subgroups. That is, if $P'$ is another $p$-subgroup, then the Sylow $p$-subgroup will contain $P'$.
\end{defi}


It follows from Lagrange's Theorem that if $p^{e}$ is the largest power of $p$ dividing the order of a group $G$, then a subgroup of $G$ of order $p^{e}$ is a maximal $p$-subgroup. It is not clear though that $G$ has any subgroups of order $p^{e}$, but it is clear that the maximal $p$-subgroup always exists. This maximality means that if $Q$ is another $p$-subgroup of $G$ such that $P\subseteq Q$, then $P=Q$.

Let us show that if $S$ is any $p$-subgroup of $G$ (perhaps $S = \{1\}$), then there exists a Sylow $p$-subgroup $P$ containing $S$. If there is no $p$-subgroup strictly containing $S$, then $S$ itself is a Sylow $p$-subgroup. Otherwise, there is a $p$-subgroup $P_1$ with $S\varsubsetneq P_1$. If $P_1$ is maximal, it is Sylow, and we are done. Otherwise there is some $p$-subgroup $P_2$ with $P_1\varsubsetneq P_2$. This procedure of producing larger and larger $p$-subgroups $P_i$ must end after a finite number of steps because by hypothesis $|G|<\infty$ and $|P_i|<|G|$ for all $i$; the largest $P_i$ must, therefore, be a Sylow $p$-subgroup.

\begin{lema}
Let $P$ be a Sylow $p$-subgroup of a finite group $G$.
\begin{enumerate}[(i)]
    \item Every conjugate of $P$ is also a Sylow $p$-subgroup of $G$.
    \item $|N_G(P)/P|$ is prime to $p$.
    \item If $a\in G$ has order some power of $p$ and $aPa^{-1} = P$, then $a\in P$.
\end{enumerate}
\end{lema}
\begin{proof}
\begin{enumerate}[(i)]
    \item Suppose $a\in G$ and $aPa^{-1}$ not a Sylow $p$-subgroup of $G$. Then there exists a $p$-subgroup greater, i.e. $Q$ with $aPa^{-1}\varsubsetneq Q$. But then $P\varsubsetneq a^{-1}Qa$ contradicting that $P$ is a Sylow $p$-subgroup of $G$.
    
    \item If $p$ divides $|N_G(P)/P|$, then Cauchy's Theorem shows that $N_G(P)/P$ contains an element $aP$ of order $p$, and hence $N_G(P)/P$ contains a subgroup $S^*=\langle aP\rangle$ of order $p$. By the Correspondence Theorem (\ref{prop:correspondencetheorem}), there is a subgroup $S$ with $P\subseteq S\subseteq N_G(P)$ such that $S/P\cong S^*$. But $S$ is a $p$-subgroup of $N_G(P)\subseteq G$ strictly larger than $P$, contradicting the maximality of $P$. We conclude that $p$ does not divide $|N_G(P)/P|$.
    
    \item By the definition of normalizer, the element $a$ lies in $N_G(P)$. If $a\not\in P$, then the coset $aP$ is a nontrivial element of $N_G(P)/P$ having order some power of $p$; inlight of part (ii), this contradicts Lagrange's Theorem.
\end{enumerate}
\end{proof}


Now I will state the three Sylow Theorems all at once, and I will not change anything from what I got from the classes of ``Estructures Algebraiques''. Again, I will not provide the proofs because they can be found in my old notes and also in the references \cite{rotmanadvancedmodernalgebra}, \cite{hungerfordalgebra} and \cite{dummitfooteabstractalgebra}. 

\begin{ter}[Sylow]\index{Sylow theorems}\label{ter:sylow}
\begin{enumerate}[(1)]
    \item Let $G$ be a finite group, $p$ a prime number and $r>0$ an integer such that $p^r\mid |G|$. Then there exist $H_1,\ldots,H_r$ subgroups of $G$ such that
    \begin{enumerate}[(i)]
        \item $|H_i| = p^{i}$, for all $1\leq i\leq r$, and
        \item $H_i\vartriangleleft H_{i+1}$, for all $1\leq i\leq r-1$.
    \end{enumerate}
    
    \item Let $G$ be a finite group, $H$ a $p$-subgroup of $G$ and $S$ a Sylow $p$-subgroup of $G$. Then, there exists $x\in G$ such that $H\subset xSx^{-1}$. In particular, every Sylow $p$-subgroup is conjugate to $S$. This also means that all Sylow $p$-subgroups are isomorphic (with same $p$).
    
    \item Let $G$ be a group and $n_p$ the number of Sylow $p$-subgroups of $G$. Then it is satisfied
    \begin{enumerate}[(i)]
        \item $n_p = [G:N_G(S_p)]$, for all $S_p$ Sylow $p$-subgroup of $G$;
        \item $n_p\mid [G:S_p]$, for all Sylow $p$-subgroup $S_p$;
        \item $n_p\equiv 1$ mod $p$.
    \end{enumerate}
\end{enumerate}
\end{ter}


From these theorems it follows some further results that can be useful.

\begin{coro}
A finite group $G$ has a unique Sylow $p$-subgroup $S$ for some prime $p$ if and only if $S\vartriangleleft G$.
\end{coro}




\end{document}