\documentclass[../main.tex]{subfiles}



\begin{document}


\section{Cosets and Lagrange's Theorem}
In this section we obtain the first significant theorem relating the structure of a finite group $G$ with the number theoretic properties of its order $|G|$. We begin by extending the concept of congruence modulo $m$ in the group $\mathbb{Z}$. By definition, $a\equiv b$ (mod $m$) if and only if $m\mid a-b$, that is, if and only if $a-b$ is an element of the subgroup $\langle m\rangle = m\mathbb{Z} = \{mk\;:\;k\in\mathbb{Z}\}$ (see cyclic groups). The next definition will generalize this concept to any group.

\begin{defi}
[Congruent]\index{Right congruent}\index{Left congruent}\index{Congruent} Let $H$ be a subgroup of a group $G$ and $a,b\in G$. We say $a$ is \textit{right congruent} to $b$ modulo $H$, denoted $a\equiv_r b$ (mod $H$) if $ab^{-1}\in H$. Analogously, $a$ is \textit{left congruent} to $b$ modulo $H$, denoted $a\equiv_\ell b$ (mod $H$), if $a^{-1}b\in H$.
\end{defi}

\begin{nota}
If $G$ is abelian, then right and left congruence modulo $H$ coincide, since $ab^{-1}\in H\Leftrightarrow (ab^{-1})^{-1}) = ba^{-1} = a^{-1}b\in H$. There also exist non-abelian groups $G$ and subgroups $H$ such that right and left congruence coincide, but this is not true in general.
\end{nota}

\begin{ter}
\label{ter:congruenceclasses} Let $H$ be a subgroup of $G$.
\begin{enumerate}[(i)]
    \item Right (resp. left) congruence modulo $H$ is an equivalent relation on $G$.
    \item The equivalence class of $a\in G$ under right (resp. left) congruence modulo $H$ is the set $Ha = \{ha\;:\;h\in H\}$ (resp. $aH = \{ah\;:\;h\in H\}$. This is called the \textit{right coset}\index{Right coset} (resp. \textit{left coset})\index{Left coset}.
    \item $|aH| = |Ha| = |H|$, for all $a\in G$.
\end{enumerate}
\end{ter}
\begin{proof}
\noproof
\end{proof}

\begin{coro}
Let $H$ be a subgroup of a group $G$.
\begin{enumerate}[(a)]
    \item $G$ is the union of the right (resp. left) cosets of $H$ in $G$.
    \item Two right (resp. left) cosets of $H$ in $G$ are either disjointed or equal.
    \item For all $a,b\in G$, $Ha = Hb\Leftrightarrow ab^{-1}\in H$ and $aH = bH\Leftrightarrow a^{-1}b\in H$.
    \item If $\mathcal{R}$ is the set of distinct right cosets of $H$ in $G$ and $\mathcal{L}$ is the set of distinct left cosets of $H$ in $G$, then $|\mathcal{R}| = |\mathcal{L}|$.
\end{enumerate}
\end{coro}



\begin{defi}
[Index]\index{Index} Let $H$ be a subgroup of $G$. The \textit{index of $H$ in $G$}, denoted $[G:H]$, is the cardinal number of the set of distinct right (resp. left) cosets of $H$ in $G$.
\end{defi}

\begin{ter}[Lagrange]\index{Lagrange's theorem}\label{ter:lagrange}
Let $G$ be a group and $H$ a subgroup of $G$. Then $G$ is finite if, and only if, $H$ and $[G:H]$ are finite. In this case $|G| = |H|\cdotp[G:H]$.
\end{ter}
\begin{proof}
\noproof
\end{proof}


To end this section, I will talk about permutation groups, cyclic groups and I will give other examples of finite groups, that may be used later in this text. I will present them as something already known, so I will not stop myself to explain any details about them.

Recall that, if we have a group $G$ with its associated product $\bullet$, then we may write $x^n$, where $x\in G$, to indicate that $x$ is operated with himself $n$ times within the product $\bullet$. If $G$ is a additive group, one may find $nx$ to indicate the same, as in additive groups we use $+$ or similar symbols to indicate the operation. 

\begin{defi}
[Order of an element]\index{Order of an element} Let $G$ be a group and $x\in G$ an arbitrairy element. Then we call \textit{order of $x$}, denoted by $\mathrm{ord}(x)$, to the lowest positive integer $m$ such that $x^m = e$, where $e$ denotes the neutral element of $G$. If $m = 0$ we say that $\mathrm{ord}(x) = \infty$.
\end{defi}

\begin{defi}
[Subgroup generated by an element]\index{Subgroup generated by an element} Let $G$ be a group and $x\in G$ be an arbitrary element. Then we call the \textit{subgroup of $G$ generated by $x$}, denoted by $\langle x\rangle$, to the set
\begin{equation}
    \notag
    \langle x\rangle := \{x^n\;:\;n\in\mathbb{Z}\}
\end{equation}
\end{defi}

The Laws of Exponents show that $\langle x\rangle$ is, in fact, a subgroup: $1 = x^0\in\langle x\rangle$, $x^nx^m = x^{n+m}\in\langle x\rangle$ and $x^{-1}\in\langle x\rangle$.

\begin{defi}
[Cyclic group]\index{Cyclic group} A group $G$ is called \textit{cyclic} if there exists $x\in G$ such that $G = \langle x\rangle$. In this case, we call $x$ a \textit{generator} of the group $G$.
\end{defi}

\begin{prop}
Let $G$ be a group. If $x\in G$, then the order of $x$ is equal to $|\langle x\rangle|$, the order of the cyclic subgroup generated by $x$.
\end{prop}
\begin{proof}
\noproof
\end{proof}

Recall from Arithmetic that, taking $n\geq 1$ integer, we defined the \textit{Euler $\phi$-function} as 
\begin{equation}
    \notag
    \phi(n) := \#\{k\in\mathbb{Z}\;:\;1\leq k\leq n\;\text{and}\;\gcd(k,n) = 1\}
\end{equation}

\begin{ter}
\begin{enumerate}[(i)]
    \item If $G = \langle a\rangle$ is a cyclic group of order $n$, then $a^k$ is a generator of $G$ if and only if $\gcd(k,n) = 1$.
    \item If $G$ is a cyclic group of order $n$ and call $g(G)$ the set of all generators of $g$, then $\#g(G) = \phi(n)$, where $\phi(n)$ is the Euler $\phi$-function.
\end{enumerate}
\end{ter}
\begin{proof}
\noproof
\end{proof}

\begin{ej}
Let's check out some examples. 
\begin{enumerate}[(i)]
    \item The group $S_3$ is not cyclic, for $\tau_1,\tau_2,\tau_3$ are transpositions (i.e. of order 2) and $\sigma_1,\sigma_2$ have order 3.
    \item The group $\mu_n = \{z\in\mathbb{C}^*\;:\;z^n = 1\}$, i.e. the group of all $n$-th roots of unity over $\mathbb{C}$, is a cyclic group generated by $e^{2\pi i/n}\}$. This is an important group regarding solutions of polynomials over an algebraically closed field, but in this notes I will not study it so much.
\end{enumerate}
\end{ej}

\begin{coro}
\label{coro:primeorderiscyclic} Every group $G$ of order $p$ prime is cyclic.
\end{coro}
\begin{proof}
Take $x\in G$ such that $x\not=e$, where $e$ is the neutral element of $G$. Then, its order must be strictly higher than 1 and also has to divide $p$, the order of $G$. As $p$ is prime, it must be $\mathrm{ord}(x) = p$, hence, by previous results, it must be $G = \langle x\rangle$
\end{proof}

\begin{prop}
\label{prop:cyclicintegers} 
\begin{enumerate}[(i)]
    \item All cyclic infinite group is isomorphic to $\mathbb{Z}$ with $+$ operation.
    \item All cyclic finite group of order $m$, with $m\in\mathbb{Z}_{>0}$, is isomorphic to $\mathbb{Z}/m\mathbb{Z}$ with $+$ operation.
\end{enumerate}
\end{prop}
\begin{proof}
The proof is rather easy considering the map, taking $x\in G$ such as $G = \langle x\rangle$, defined by $f_x:\mathbb{Z}\rightarrow \langle x \rangle =G$, $n\mapsto x^n$. This is an epimorphism, which can be easily proved, and then we specify each case. If $G$ is infinite, only $0\mapsto e_G$, and then $\ker f_x = \{0\}$ because no other element makes $x^m = e_G$ rather than $m = 0$, and thus $f_x$ is an isomorphism. The other case is considering that $G$ has order $m<\infty$, i.e. $x$ has order $m$. Then, $\ker f_x = m\mathbb{Z}$ and we can use the First Isomorphism Theorem (which I state later, at \ref{ter:firstisomorphism}) to prove the isomorphism $\mathbb{Z}/m\mathbb{Z}\cong G$.
\end{proof}

Indeed, this is cheating because I know the Isomorphism Theorems but I still have not stated them in this notes. But the purpose of these notes is just to collect information about Groups and hence I do not really care about the order.

To end the subsection talking about cyclic groups, I will state a proposition that collects some random facts about cyclic groups that can be useful in the future.

\begin{prop}
\label{prop:cyclicgroupsrandom} \begin{enumerate}[(1)]
    \item Let $G_1$ and $G_2$ be cyclic groups. Then, $|G_1| = |G_2|\Longleftrightarrow G_1\cong G_2$.
    \item If $G$ is a cyclic group, then all subgroup of $G$ is also cyclic.
    \item If $G$ is a cyclic group of order $n$, then for all $d$ divisor of $n$ there exists a unique subgroup of order $d$ of $G$.
\end{enumerate}
\end{prop}

As I will be interested in the future, I will define now a special kind of group: the Klein Group or the Four-Group. Also there will be some new concepts as the cartesian product of groups which I did not introduce, because I assume they are already known and are not so important to the purpose of these notes.

\begin{defi}
[Klein Group] The \textit{Klein Group}\index{Klein Group} or \textit{Four-Group}\index{Four-Group} is a group with only four elements. It is often written $\mathbb{V}_4$ and it is isomorphic to $C_2\times C_2$, where $C_2$ means the cyclic group of order 2\footnote{See that there is only a group of order 2, for \ref{prop:cyclicgroupsrandom} point 1.} A presentation of the group can be 
\begin{equation}
    \notag
    \mathbb{V}_4 = \{a,b\;:\;a^2=b^2=(ab)^2=e\} = \{a,b,ab,e\}
\end{equation}
\end{defi}

Observe that there are two groups of order 4. One of these is $\mathbb{V}_4\cong C_2\times C_2\cong\mathbb{Z}/2\mathbb{Z}\times\mathbb{Z}/2\mathbb{Z}$ and the other is $C_4\cong \mathbb{Z}/4\mathbb{Z}$. In Klein group, there are only three elements plus the neutral element, and every element can be obtained by multiplying the other two. Also it is an abelian group.



To end this section, I give a brief description of the symmetric group and permutation groups.

Given a set $X$, a permutation of $X$ is a map from $X$ to $X$. The set $S_X$ with the composition of maps is a group with the identity map as neutral element. For an integer $n\geq 1$, we call permutation group to the group of all permutations of $\{1,\ldots,n\}$, technically written $S_{\{1,\ldots,n\}}$ but to simplify we will write $S_n$. Then $S_n$ is the group of all possible permutations of $n$ elements. We call it the \textit{symmetric group of degree $n$}\index{Symmetric group of degree $n$}. We denote its elements $\sigma\in S_n$ by a matrix indicating which element is changed by which:
\begin{equation}
    \notag
    \sigma = 
    \begin{pmatrix}
    1 & 2 & 3 & \cdots \\
    \sigma(1) & \sigma(2) & \sigma(3) & \cdots
    \end{pmatrix}
\end{equation}
It is clear to see that $|S_n| = n!$ and also that if $\sigma,\tau\in S_n$, then $\sigma\tau$ (as composition) satisfies that $(\sigma\tau)(i) = \sigma(\tau(i))$, for all $i\in\{1,\ldots,n\}$.

Moreover, given $r$ different elements component-wise $k_1,\ldots,k_r$ from $\{1,\ldots,n\}$, we write $(k_1,\ldots,k_r)$ the permutation $\sigma$ from $S_n$ defined by
\begin{equation}
    \notag
    \left\{
    \begin{array}{ll}
        \sigma(k_1) = k_2,\;\sigma(k_2) = k_3,\;\cdots\;\sigma(k_{r-1}) = k_r,\;\sigma(k_r) = k_1\\
        \sigma(p) = p,\;\;\forall p\in\{1,\ldots,n\}\setminus\{k_1,\ldots,k_r\}
    \end{array}
    \right.
\end{equation}
and we call $\sigma$ an $r$-cycle.

For an example, take $S_3$. Then we can write its elements as $\tau_1 = (2,3)$, $\tau_2=(1,3)$ and $\tau_3 = (1,2)$, its 2-cycles, and also $\sigma_1 = (1,2,3)$ and $\sigma_2 = (1,3,2)$ its 3-cycles, together with the identity. In general, we don't have a unique way of representing cycles. For example, $(1,2,3) = (2,3,1) = (3,1,2)$. We also call \textit{transposition}\index{Transposition} to a 2-cycle and then we have the following important statements:

\begin{prop}\label{prop:symmetricgroup} Let $n\geq 1$ be an integer.
\begin{enumerate}[(i)]
    \item If $\tau\in S_n$ is a transposition, then $\tau^2 = \mathrm{id}$, i.e. a transposition is of order 2, i.e. the inverse of a transposition is itself.
    \item All permutation from $S_n$, different from identity, can be written as a disjoint cycle product unequivocally determined except from the order. 
    \item All cycles can be written as a product of transpositions.
    \item All permutations can be written as a product of transpositions
\end{enumerate}
\end{prop}
\begin{proof}
The first statement is more an observation, as it is very trivial. The second one is not trivial, but I leave its proof as an exercise. The third one is very easy to prove and the last one is a direct consequence of the second and third statements.
\end{proof}

Regarding this statement, we call the \textit{signature}\index{Signature of a permutation} of a permutation $\sigma\in S_n$, to the number $(-1)^t$, where $t\in \mathbb{Z}_{\geq 0}$ is the number of transpositions in which $\sigma$ decomposes. Sometimes it is written $\varepsilon(\sigma)$, sometimes $\mathrm{sg}(\sigma)$. We say that $\sigma$ is an \textit{even permutation}\index{Even permutation} if $\varepsilon(\sigma) = 1$ and an \textit{odd permutation}\index{Odd permutation} if $\varepsilon(\sigma) = -1$.

\begin{defi}
[Alternate group] We call the \textit{alternate group}\index{Alternate group} to the subgroup of $S_n$ which contains all the even permutations. It is written
\begin{equation}
    \notag
    A_n:=\{\sigma\in S_n\;:\;\varepsilon(\sigma) = 1\}
\end{equation}
\end{defi}


\end{document}