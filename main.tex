\documentclass[titlepage, main=catalan, 11pt]{book}
\renewcommand{\baselinestretch}{1.2}
\usepackage[utf8]{inputenc}
\usepackage{amsfonts}
\usepackage{amsmath}
\usepackage{tikz}
\usepackage{pgfplots}
\usepackage{color}
\usepackage{biblatex}
\usepackage{graphicx}
\usepackage{vmargin}
\usepackage{enumerate}
\usepackage{cancel}
\usepackage{amssymb}
\usepackage{amsthm}
\usepackage[catalan]{babel}
\usepackage{mathrsfs}
\usepackage[colorlinks=true,linkcolor=blue]{hyperref}
\usepackage{float}
\usepackage{lscape}
\usepackage{multirow}
\usepackage[all]{xy}
\usepackage{makeidx}
\usepackage{lipsum}
%\usepackage[T1]{fontenc}
\usepackage{comment}
\usepackage{transparent}
\usepackage{textcomp} %podem escriure ``o'' amb la comanda \textdegree també € amb \texteuro
\usepackage[gen]{eurosym} %tb official en lloc de ``gen''
\usepackage[section]{placeins} %Evita que les figures saltin de secció
\usepackage{fancyref}
\usepackage{array}
\usepackage{longtable}
\usepackage{mathtools}
\usepackage{commath}
\usepackage{stmaryrd}

\usepackage{scrextend}%Indentar un bloc sencer
%\usepackage{tgpagella} % Para la letra guay
\usepackage{dtk-logos}
\addbibresource{AAbibliografia.bibtex}

\usepackage{blindtext}
\usepackage{multicol}

\usepackage{subfiles} % Best loaded last in the preamble
\usepackage{pst-all}
\usepackage{qtree}





%   MÁRGENES
\setmargins
{1.5cm}                     % margen izquierdo
{1cm}                     % margen superior
{18cm}                    % anchura del texto
{25cm}                   % altura del texto
{8pt}                      % altura de los encabezados
{0.8cm}                       % espacio entre el texto y los encabezados
{0pt}                       % altura del pie de página
{2cm}                       % espacio entre el texto y el pie de página






%   Para enumerar las definiciones y demás ponemos 
%   \newtheorem{abreviatura}{Nombre}[chapter]. Donde:
%       abreviatura es el nombre que le damos al entorno
%       Nombre es la palabra que aparecerá en negrita, junto con el número
%       Chapter si queremos que haga la numeración rollo 1.4 y eso
%   https://www.emis.de/journals/RCE/IntroLatex/Curso%20LaTeX%208.pdf

\theoremstyle{definition}
    \newtheorem{defi}{Definició}[section] %   DEFINICIONES

    \newtheorem{ej}[defi]{Exemple}   %   EJEMPLOS    quiero que la numeración sea independiente
    \newtheorem{alg}{Agorisme}

    
\theoremstyle{plain}
    \newtheorem{prop}[defi]{Proposició}   %   PROPOSICIONES para que la numeración sea correlativa a la de defi, ponemos [defi]
    \newtheorem{lema}[defi]{Lema}
    \newtheorem{coro}[defi]{Corol·lari}
    \newtheorem{ter}[defi]{Teorema}
    \newtheorem{exercici}{Exercici}
    \newtheorem*{rnd}{Continuació}
    \newtheorem{nota}[defi]{Observació}    %   NOTAS
    \newtheorem{rec}[defi]{Recordatori}     %   RECORDATORIS

\theoremstyle{remark}
    \newtheorem*{sol}{Solució}
    
    
    
    
\newcommand{\mcd}{\text{mcd}}
\newcommand{\mcm}{\text{mcm}}
\newcommand{\gr}{\text{gr}\;}
\newcommand{\Irr}{\text{Irr}}
\newcommand{\Ima}{\text{Im}}
\newcommand{\Gal}{\text{Gal}}
\newcommand{\Aut}{\text{Aut}}
\newcommand{\F}[1]{\mathbb{F}_{#1}}
\newcommand{\R}{\mathbb{R}}
\newcommand{\Z}{\mathbb{Z}}
\newcommand{\elevadoam}[1]{#1^{\mathcal{M}}}
\newcommand{\sust}[2]{\left(\begin{smallmatrix}#1\\#2\end{smallmatrix}\right)}
\newcommand{\ob}[1]{\mathrm{Ob}(\mathcal{#1})}
%\newcommand{\hom}[1]{\mathrm{Hom}_{\mathcal{#1}}}

\newcommand{\interpretami}[2]{#1^{\mathcal{M}_{#2}}}

\newcommand{\noproof}{I will not do this proof as it migth be found on my ``Estructures Algebraiques'' course's notes and it is not necessary for the purpose of this text.}
    






    
    
    





\title{Lógica proposicional}
\author{Víctor Santiago Blanco}
\date{Tardor 2018}

\setpapersize{A4}
\makeindex





















\pgfplotsset{compat=1.15}
\begin{document}    %%%%%%%%%%%%%%%%%%%%%%%%%%%%%%%%%%%%%%%%%%%%%%%%%%%%%%%%55





\begin{titlepage}
    \centering
    {\bfseries\LARGE Universitat de Barcelona \par}
    \vspace{1cm}
    {\scshape\Large Facultat de Matemàtiques \par}
    \vspace{3cm}
    {\scshape\Huge Topologia \par}
    \vspace{3cm}
    {\itshape\Large Apunts de teoria i exercicis \par}
    \vfill
    {\Large Autor: \par}
    {\Large Víctor Santiago Blanco \par}
    \vfill
    {\Large 2020 - 2021 \par}
    \end{titlepage}

\newpage



\tableofcontents
\newpage







\setcounter{section}{-1}

\chapter*{Introducció}

En el present document es troba la gran compilació d'apunts de totes les assignatures de topologia que he realitzat a la carrera. En total he realitzat tres assignatures relacionades amb la topologia: una anomenada \textit{Topologia} que la vaig cursar a tercer de carrera, essent l'assignatura de segon en realitat, i durant el primer semestre de pandèmia, amb la qual cosa va ser una mica caòtica la docència i, per tant, és una assignatura que em vaig treure bastant autodidàcticament; la segona s'anomena \textit{Topologia i Geometria Diferencial de Superfícies}, que constava de dues parts, la primera de les quals referida a la part de topologia algebraica que parla sobre el grup fonamental, que és la que incloc aquí. La tercera és una optativa que, a dia d'avui 17 de gener de 2022 estic cursant i, de fet, demà tinc l'examen final de l'assignatura. S'anomena \textit{Topologia Algebraica} i consisteix en l'estudi de l'homologia simplicial i singular.

Aquestes notes no són gens rigoroses i no han estat pròpiament revisades, amb la qual cosa estaran plenes d'errors tipogràfics i de contingut, però ja em serveixen per a l'estudi. Cada part té la seva introducció amb la qual cosa ja no hi ha res més a dir aquí.









\part{Topologia}
\subfile{topo1/topo1.tex}
\part{Grup fonamental}
\subfile{topo2/topo2.tex}
\part{Grups d'homologia}
\subfile{topo3/topo3.tex}


\part{Apèndix}
\appendix

\subfile{appendix/appendix.tex}




\printindex
\printbibliography
\end{document}
