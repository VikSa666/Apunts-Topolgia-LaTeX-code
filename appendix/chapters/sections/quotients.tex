\documentclass[../main.tex]{subfiles}



\begin{document}


\section{Normality, quotient groups and isomorphism's theorems}

In the last section we described the cosets, i.e. the sets of left and right congruent classes. Now we want to describe and structure as a group the quotient set formed by a group $G$ and a subgroup $H$ of $G$. Our first problem is that we should give two different definitions, one for the right cosets and the other for the left cosets. The second problem is that the product of classes would depend on the representant of the class, i.e. it would not be well defined. All these problems will be solved with the new concept we are going to treat right now: the normal subgroup. That is, we will not be able to construct the quotient for any group and any subgroup, we will need a special subgroup.

\begin{defi}
[Normal subgroup]\index{Normal subgroup} A subgroup $H\subset G$ of a group $G$ is said to be \textit{normal in $G$}, denoted $H\vartriangleleft G$, if $\forall x\in G$ it occurs one of the following (which are equivalent):
\begin{enumerate}[(i)]
    \item $xHx^{-1}\subseteq H$, where $xHx^{-1} = \{xhx^{-1}\;:\;h\in H\}$, or
    \item $xHx^{-1} = H$, or
    \item $xH = Hx$.
\end{enumerate}
\end{defi}

Before continuing to the next statement, I will define two different product of groups, as they will be useful from now on. The first one is the direct product which is the same definition for any given two arbitrary sets. The second one is a proper product of groups.

First, let's see a new important group. 

\begin{defi}
If $H$ and $K$ are subgroups of a group $G$, then
\begin{equation}
    \notag
    H\vee K = \langle H\cup K\rangle
\end{equation}
is the \textit{subgroup generated by $H$ and $K$}\index{Subgroup generated by two groups}\index{$H\vee K$}.
\end{defi}

This is the smallest subgroup of $G$ that contains both $H$ and $K$. It is easy to check that. Also, if $G$ is abelian, then $H\vee K = H+K = \{sh+tk\;:\;h\in H, k\in K,\;s,t\in\mathbb{Z}\}$. This is not a real product, but the notation reminds me of it, so that is why I give this definition now, talking about products. The following product is the direct product.

\begin{defi}
[Direct product]\label{def:directproduct}\index{Direct product} Let $H$ and $K$ be groups, then their \textit{direct product}, denoted $H\times K$, is the set of all ordered pairs $(h,k)$, with $h\in H$ and $k\in K$, equipped with the operation
\begin{equation}
    \notag
    (h,k)(h',k') = (hh',kk')
\end{equation}
\end{defi}

It is easy to check that, with this definition, $H\times K$ is indeed a group with neutral element $(1_H, 1_K)$, where $1_H$ is the neutral element of $H$ and $1_K$ of $K$, and $(h,k)^{-1} = (h^{-1},k^{-1})$.

The last product I want to show is the product of all possible elements of two groups by one side. That is, if $H$ and $K$ are both groups, then we define
\begin{equation}
    \notag
    HK:=\{hk\;:\;h\in H,\;k\in K\}
\end{equation}
One may observe that, if $H$ and $K$ are subgroups of $G$, then $HK$ might not be a subgroup of $G$. For example, if $G = S_3$ and $H = \langle (1,2)\rangle$ and $K = \langle (1,3)\rangle$. Then,
\begin{equation}
    \notag
    HK = \langle \mathrm{id}, (1,2),(1,3), (1,3,2)\rangle
\end{equation}
is not a subgroup, as it is not closed because $(1,3)(1,2) = (1,2,3)\not\in HK$.

With these definitions we can continue our discussion on normal subgroups and its properties, regarding these new structures we just saw.

\begin{prop}
Let $K$ and $N$ be subgroups of a group $G$, with $N$ normal in $G$. Then,
\begin{enumerate}[(a)]
    \item $N\cap K$ is a normal subgroup of $K$;
    \item $N$ is a normal subgroup of $N\vee K$;
    \item $NK = N\vee K = KN$;
    \item if $K$ is normal in $G$ and $K\cap N = \{e\}$, then $nk = kn$ for all $k\in K$ and $n\in N$.
\end{enumerate}
\end{prop}
\begin{proof}
\noproof    
\end{proof}


Now finally we can define the quotient group and give to it a group structure. First we must define the quotient operation.

\begin{defi}[Quotient group]
Let $G$ a group and $H$ a subgroup of $G$. Consider the binary operation $\bullet_q$ defined as follows:
\begin{equation}
    \notag
    (xH)\bullet_q(yH) :=\{k_1k_2\;:\;k_1\in xH,\;k_2\in yH\}
\end{equation}
If $H$ is normal on $G$ the $\bullet_q$ operation is $(xH)\bullet_q(yH) = (xy)H$ (the proof is left as an exercise). If we denote $G/H$, with $H$ normal on $G$, as the set of all left (or right) cosets of $G$ then $(G/H,\bullet_q)$ is a group. It is called the \textit{quotient group}\index{Quotient group}.
\end{defi}

It remains as an exercise to prove that this quotient group is, indeed, a group. One must prove that the operation is well defined, i.e. it does not depend on the representant, and then prove that it has a group structure following the definition. We will denote often the elements of $G/H$ as $[x]$ to denote it is the class of $x\in G$ modulo $H$. As it will be the same, given that $H\vartriangleleft G$, we will have $[x] = xH = Hx$, $\forall x\in G$.

\begin{defi}
[Quotient map]\index{Quotient map}\index{Natural map} Let $G$ be a group and $H$ be a normal subgroup of $G$. Define the \textit{natural map} or \textit{quotient map} $\pi:G\rightarrow G/H$ by $\pi(x) = [x]$, i.e. the projection to the quotient.
\end{defi}

\begin{nota}
It is clear that $\pi$ is a homomorphism of groups and also that $\ker\pi = \{a\in G\;:\;\pi(a) = H\} = \{a\in G\;:\;[a] = H\} = H$, i.e. if $y\in H$ then $\pi(y) = [e] = 1_{G/H}$. It is also important to remark that $\pi$ is an epimorphism.
\end{nota}


\begin{ter}
Let $G$ and $G'$ be groups, and let $H$ be a subgroup of $G$. Then, $H$ is normal on $G$ if, and only if, there exists some homomorphism $f:G\rightarrow G'$ such that $\ker f = H$.
\end{ter}
\begin{proof}
\noproof
\end{proof}


\begin{prop}
Let $f:G\rightarrow G'$ a group homomorphism.
\begin{enumerate}[(a)]
    \item If $H$ is a subgroup of $G$, then $f(H)$ is a subgroup of $G'$.
    \item If $H'$ is a subgroup of $G'$, then $f^{-1}(H')$ is a subgroup of $G$. Moreover, if $H'$ is a normal subgroup of $G'$, then $f^{-1}(H')$ is a normal subgroup of $G$.
    \item If $H_1\vartriangleleft G_1$, then $f(H_1)\vartriangleleft f(G_1)$ but not necessarily in $G_2$.
\end{enumerate}
\end{prop}



Let $G,G'$ be groups, $f:G\rightarrow G'$ a group homomorphism and let $H$ be a subgroup normal on $G$. We say that $f$ \textit{factorizes through $G/H$} if there exists a group homomorphism $\overline{f}:G/H\rightarrow G'$ such that $f = \overline{f}\circ \pi$, where $\pi:G\rightarrow G/H$ is the natural homomorphism, i.e. if there exists a group homomorphism that makes the following diagram commutative:
\begin{equation}
    \notag
    \xymatrix{
    G \ar[r]^f \ar[dr]_\pi & G' \\
    & G/H\ar[u]_{\overline{f}}
    }
\end{equation}

Firstly, I will state a result that will be useful to demonstrate the Isomorphism theorems and then I will state these theorems.

\begin{prop}\label{prop:isomorphismtheorem}
Let $G$ and $G'$ be groups and $f:G\rightarrow G'$ a homomorphism, and let $H$ be a normal subgroup of $G$. Then, $f$ factorizes through $G/H$ if, and only if, $H\subset \ker f$.
\end{prop}
\begin{proof}
Take the implication to the right. Take as hypothesis that $f$ factorizes through $G/H$. Let $h\in H$. According to the definition of $\pi$ and that $\overline{f}$ is a homomorphism, we obtain that $f(h) = \overline{f}(\pi(h)) = \overline{f}([e]) = e'$, where $[e]$ means the neutral element of $G/H$ and $e'$ the neutral element of $G'$. Then, it is clear that $H\subseteq \ker f$.

Take the implication to the left. We take as hypothesis that $H\subset \ker f$. We define now $\overline{f}:G/H\rightarrow G'$ as $\overline{f}([x]) = f(x)$, where $[x]$ indicates the class of $x\in G$. We have to see that this is well-defined, i.e. that the definition does not depend on the representant chosen for the class. If $y\in [x]$, we have $y = xh$, for $h\in H$. Now, this means that
\begin{equation}
    \notag
    f(y) = f(xh) = f(x)f(h) = f(x)e' = f(x)
\end{equation}
because $h\in H\subset \ker f$. Now, if $x,y\in G$, we have $\overline{f}([x][y]) = \overline{f}([xy]) = f(xy) = f(x)f(y) = \overline{f}([x])\overline{f}([y])$ because of the definition of the class-product. Hence, $\overline{f}$ is a homomorphism. Eventually, it is clear that $f = \overline{f}\circ \pi$.
\end{proof}

\begin{ter}
[First isomorphism]\label{ter:firstisomorphism}\index{First isomorphism theorem} Let $G$ and $G'$ be groups and $f:G\rightarrow G'$ a group homomorphism. Then, $f$ factorizes through $G/\ker f$ and we have $f = i\circ \Tilde{f}\circ \pi$, where $\Tilde{f}$ is a group isomorphism from $G/\ker f$ to $\mathrm{Im}f$, $\pi:G\rightarrow G/H$ is the normal morphism and $i:\mathrm{Im}f \rightarrow G$ is the inclusion. We have, then, that the following diagram is commutative:
\begin{equation}
        \notag
        \xymatrix{
            G\ar[dr]_{\pi} \ar[r]^{f} & G'\\
            & G/H \ar[u]_{\Tilde{f}}
        }
\end{equation}
\end{ter}
\begin{proof}
Because of the last proposition \ref{prop:isomorphismtheorem}, there exists a homomorphism $\overline{f}:G/\ker f\rightarrow G'$ such that $f = \overline{f}\circ\pi$. Clearly $\overline{f}$ is injective and $\overline{f} = i\circ\Tilde{f}$, where $\Tilde{f}$ is an isomorphism from $G/\ker f$ into $\mathrm{Im}\overline{f}$. Finally, as $\mathrm{Im}\overline{f} = \mathrm{Im}f$ by the definition of $\Tilde{f}$, we obtain hence the result.
\end{proof}

Now, for the following two theorems, I will just state the results, but I won't prove them, because I think it is not necessary for the purpose of these notes. One can find the proofs in any books and also in my last Group Theory notes from the classes of \textit{Estructures Algebraiques}.

\begin{ter}
[Second isomorphism]\label{ter:secondisomorphism}\index{Second isomorphism theorem} Let $G$ be a group, $H$ a normal subgroup and $F$ an arbitrary subgroup of $G$. Then, $HF$ is subgroup of $G$, $F\cap H$ is normal subgroup of $F$ and $H$ is normal subgroup of $HF$. Moreover, the inclusion $F\hookrightarrow HF$ induces an isomorphism between $F/(F\cap H)\cong (HF)/H$.
\end{ter}

\begin{ter}
[Third isomorphism]\label{ter:thirdisomorphism}\index{Third isomorphism theorem} Let $\varphi:G\rightarrow G'$ an epimorphism, $H'$ a normal subgroup of $G'$ and $H:=\varphi^{-1}(H')$. Then $\varphi$ induces an isomorphism between $G/H$ and $G'/H'$.
\end{ter}


\begin{prop}
[Correspondence Theorem]\label{prop:correspondencetheorem} Let $G$ be a group, let $K\vartriangleleft G$, and let $\pi:G\rightarrow G/K$ be the natural map. Then
\begin{equation}
    \notag
    S\mapsto\pi(S) = S/K
\end{equation}
is a bijection between $\mathrm{Sub}(G;K)$, the family of all those subgroups $S$ of $G$ that contain $K$, and $\mathrm{Sub}(G/K)$, the family of all the subgroups of $G/K$. Moreover, $T\subseteq S\subseteq G$ if, and only if, $T/J\subseteq S/K$, in which case $[S:T] = [S/K : T/K]$, and $T\vartriangleleft S$ if and only if $T/K\vartriangleleft S/K$, in which case $S/T\cong (S/K)/(T/K)$.
\end{prop}
\begin{proof}
\noproof
\end{proof}

The following diagram is a way to remember this theorem:
\begin{equation}
    \notag
    \xymatrix{
    G \ar[dr]\ar@{-}[d] & \\
    S \ar[dr]\ar@{-}[d] & G/K \\
    T \ar[dr]\ar@{-}[d] & S/K \\
    K \ar[dr]\ar@{-}[d] & T/K \\
    0 & \{1\}
    }
\end{equation}





\end{document}